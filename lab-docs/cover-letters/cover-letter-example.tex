\documentclass{letter}
\usepackage[english]{babel}
\usepackage{graphicx}
\usepackage[a4paper]{geometry}
%\addtolength{\voffset}{-1.15in}
%\addtolength{\hoffset}{-3in}
%\addtolength{\textheight}{12cm}
\geometry{margin=1in}

\begin{document}

%--------------------
%	MS title
%--------------------
 
\begin{letter} {Re:  Spatio-temporal spillover risk of yellow fever in Brazil} % Article title 
%-------------------
%  UGA banner image
%-------------------

\includegraphics[width=\textwidth]{newlogo.png}

%-----------------------------------------------------------



\opening{Dear Editor, }

Please find enclosed with this letter our manuscript ``Spatio-temporal spillover risk of yellow fever in Brazil'' which we submit for publication as an article in \textit{BMC Parasites \& Vectors}. 

Our manuscript presents a statistical model which predicts the spatio-temporal spillover risk of yellow fever (YF) in each Brazilian municipality by month based on environmental and demographic covariates. Previous methods mapping yellow fever spillover risk have failed to incorporate the temporal dynamics and ecological context of the disease, and are therefore unable to predict seasonality in spatial risk across Brazil. Here, we present a temporally explicit statistical model to predict the propensity of yellow fever spillover across Brazil. We also further divide our model into a regional model consisting of two areas as determined by reservoir richness, to better contextualize the spillover process. Taking this approach, we observe that the seasonality of peak spillover risk differs across the country, giving rise to higher yellow fever spillover risk in densely populated, coastal cities than the remote Amazonian municipalities (coinciding with yellow fever's historical range) at certain times of the year. This counter-intuitive conclusion is particularly timely given that the most recent outbreak occurred within the coastal region, which has previously been thought to be low risk and was only recently included in the recommended vaccination area. 

These findings would be of interest to decision makers by helping prioritize the allocation of a limited number of vaccines. This work will also be of interest to the greater spatial epidemiology community. The modeling approach used to quantify vector-borne spillover is a novel application of a method originally designed to identify spillover from zoonotic reservoirs. This widening of method applicability moves the field closer to developing real-time, high resolution, risk assessment for emerging infectious diseases. 

%We believe this article is suited for this special issue because of its focus on explicitly describing the spatio-temporal transmission risk of yellow fever from non-human primate reservoirs to humans.  

These results are original and not under consideration for publication elsewhere.  All authors have approved the attached manuscript for submission. 

I can be reached by email (reni@uga.edu), phone (706-583-5538) or post (Odum School of Ecology, University of Georgia, Athens, GA 30602). Thank you for your consideration.

Sincerely,\\

RajReniKaul


\end{letter}
\end{document}


